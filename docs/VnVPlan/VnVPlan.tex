\documentclass[12pt, titlepage]{article}

\usepackage{booktabs}
\usepackage{tabularx}
\usepackage{hyperref}
\hypersetup{
    colorlinks,
    citecolor=blue,
    filecolor=black,
    linkcolor=red,
    urlcolor=blue
}
\usepackage[round]{natbib}

\let\wss\relax
%% Common Parts

\newcommand{\progname}{Problem Statement and Goals \\ Game of continuous life} % PUT YOUR PROGRAM NAME HERE
\newcommand{\authname}{Baptiste Pignier} % AUTHOR NAMES                  

\usepackage{hyperref}
    \hypersetup{colorlinks=true, linkcolor=blue, citecolor=blue, filecolor=blue,
                urlcolor=blue, unicode=false}
    \urlstyle{Same}
                                


\begin{document}

\title{System Verification and Validation Plan for \progname{}} 
\author{\authname}
\date{\today}
	
\maketitle

\pagenumbering{roman}

\section*{Revision History}

\begin{tabularx}{\textwidth}{p{3cm}p{2cm}X}
\toprule {\bf Date} & {\bf Version} & {\bf Notes}\\
\midrule
14/02/2025 & 1.0 & First release\\
\bottomrule
\end{tabularx}


\newpage

\tableofcontents

\listoftables
\wss{Remove this section if it isn't needed}

\listoffigures
\wss{Remove this section if it isn't needed}

\newpage

\section{Symbols, Abbreviations, and Acronyms}

All symbols, abbreviations and acronyms used here are the same as in the SRS \citet{SRS} ( Section 1 ). Please refer to the relevant section of the SRS for definitions. 
In this document, we will use the xSyB notation to define the generation rule used in discrete simulation. This notation means "a cell survives if and only if there are x cells in its neighborhood, and a cell is born in a cell that has y other cells in its neighborhood."

\newpage

\pagenumbering{arabic}

A Verification and Validation (V\&V) plan is a comprehensive guide outlining the goals, scope, techniques, and standards for confirming and validating a software product. 
Verification ensures that the software aligns with specified requirements and design criteria, while validation confirms that it fulfills user needs and expectations. 
This plan is crucial for maintaining software, dependability, quality, and functionality, and for identifying and resolving issues before release.

The following document presents the V\&V plan for \progname. It begins with the general information in Section 2, followed by a detailed discussion of the methods to achieve these objectives in Section 3. Finally, Section 4 provides the test descriptions.

\section{General Information}

\subsection{Summary}

The software to be tested is \progname. It can be used to simulate a continuous cellular automaton, allowing simulation parameters to be modified on the fly. 
In this way, the user can assess the effect of his modifications on the cell automaton, its evolution and its environment.

\subsection{Objectives}

The aim is to check the accuracy of the software, in the sense that the modifications applied to the simulation must be faithful to the modifications made by the user. 
Also, the graphical representation of the functions derived from the input parameters must be correct. 
For example, if the user wants the simulation to run twice as fast, then the simulation must run twice as fast.
If the user wants to add a diffusion constraint to the automaton, this constraint must be taken into account.

The quality of the code will also be assessed using static analysis such as linter.
The software depends on a graphics library, which will be assumed to be correct for the rest of the project. No checks will therefore be made on this library.

\subsection{Challenge Level and Extras}

This is a general project, as a few mathematical models are involved in developing an unpredictable simulation.
Also, particular attention will be paid to the user interface, which increases the complexity of the project.
This project does not include any extras.

\subsection{Relevant Documentation}

The SRS is crucial for VnV as it outlines what the software is supposed to do. It defines the functional and non-functional requirements that the software must meet.
During verification, we check that the software meets these requirements. Validation ensures that the software fulfills its intended use as specified in the SRS.

\citet{SRS}


\section{Plan}

This section outlines the plan for verifying the documentation and software for \progname. 
The key elements to be verified include are SRS, design, V\&V, and implementation.

\subsection{SRS Verification Plan}

The SRS \citet{SRS} will be verified via feedback from a domain expert by using the SRS checklist designed by Dr.Smith. The feedback will be collected by an issue tracker.
The author will be responsible for dealing with each issue by discussing it with the sender and applying a corrective measure or justifying its absence.

\subsection{Design Verification Plan}

\subsection{Verification and Validation Plan Verification Plan}

The V\&V Plan will be verified via feedback from a domain expert by using the V\&V checklist designed by Dr.Smith. The feedback will be collected by an issue tracker.
The author will be responsible for dealing with each issue by discussing it with the sender and applying a corrective measure or justifying its absence.

\subsection{Implementation Verification Plan}

Verification of the implementation will be carried out by manual review and application of the unit tests listed below. 
Their results will be compared with those expected and potential differences will be justified.
Automatic tests will also be used.

\subsection{Automated Testing and Verification Tools}

The automatic tests will be carried out mainly by the pylint tool, which enables static verification of the code, in particular to check that it complies with the PEP8 python programming standard.
Pylint can also be used to check good code practices such as function atomicity and execution consistency, and can be used to optimise performance.
The use of pylint will be incorporated into the project's workflow and continuous integration.

\subsection{Software Validation Plan}

As part of this project, the software will be verified by comparing the resulting simulations with those of other continuous cellular automaton simulation software.

\section{System Tests}

The following section presents the tests to which the functional and non-functional requirements will be subjected, as well as a traceability matrix.

\subsection{Tests for Functional Requirements}
		
\paragraph{Test on inputs}

\begin{enumerate}

\item{Simulation test\\}

Control: Automatic
					
Initial State: Initially, the grid will be in any state
					
Input: entries will be those corresponding to the 3B2S rule
					
Output: The output will be the simulation as it would have been with the same grid on a classic Game of Life simulator.

How test will be performed:

Inputs will be entered manually and the output will be automatically compared to that of another Game of Life simulation software through a simple comparison operation.

\item{Function test\\}

Control: Manual
					
Initial State: The function graphs will be in any state

Input: The input will be an arbitrary modification of the function parameters
					
Output: The output will be the modified graph of the function

How test will be performed:

Inputs will be entered manually, and output will be compared visually with that of another function visualization software such as Geogebra.

\end{enumerate}

\subsection{Tests for Nonfunctional Requirements}
		
\paragraph{Nonfunctional Test}

\begin{enumerate}

\item{Usability test\\}

Type: Manual
					
Initial State: The software will be launched for the first time, in an empty state.
					
Condition: Several third parties, who will never have used the software before, will be asked to run a simulation and modify its parameters, without any other help.
					
Result: The test will be passed if 80\% of the third parties successfully complete the task.
					
\item{Portability test\\}

Type: Manual
					
Initial State: The software is to be downloaded onto a machine running any operating system.
					
Condition: The software will be started and the simulation run.
					
Result : The test will be passed if the simulation runs correctly, with no problems attributable to the operating system. 
					

\end{enumerate}


\subsection{Traceability Between Test Cases and Requirements}

\begin{table}[h!]
  \centering
  \begin{tabular}{|c|c|c|c|c|}
  \hline        
    & Simulation test & Function test & Usability test &  Portability test\\
  \hline
  R1   & X &   &   & \\ \hline
  R2   & X &   &   &\\ \hline
  R4   & X &   &   & \\ \hline
  R3   &   & X &   & \\ \hline
  NFR2 &   &   & X & \\ \hline
  NFR4 &   &   &   & X\\

  \hline
  \end{tabular}
  \caption{Traceability Matrix Showing the Connections Between Test Cases and Requirements}
  \label{Table:trace}
  \end{table}

\section{Unit Test Description}

\wss{This section should not be filled in until after the MIS (detailed design
  document) has been completed.}

\wss{Reference your MIS (detailed design document) and explain your overall
philosophy for test case selection.}  

\wss{To save space and time, it may be an option to provide less detail in this section.  
For the unit tests you can potentially layout your testing strategy here.  That is, you 
can explain how tests will be selected for each module.  For instance, your test building 
approach could be test cases for each access program, including one test for normal behaviour 
and as many tests as needed for edge cases.  Rather than create the details of the input 
and output here, you could point to the unit testing code.  For this to work, you code 
needs to be well-documented, with meaningful names for all of the tests.}

\subsection{Unit Testing Scope}

As previously stated, the graphics library is assumed to be correct. Thus, no unit tests will be imposed on it. Unit tests are therefore limited to what the author has implemented.

\subsection{Tests for Functional Requirements}

\wss{Most of the verification will be through automated unit testing.  If
  appropriate specific modules can be verified by a non-testing based
  technique.  That can also be documented in this section.}

\subsubsection{Module 1}

\wss{Include a blurb here to explain why the subsections below cover the module.
  References to the MIS would be good.  You will want tests from a black box
  perspective and from a white box perspective.  Explain to the reader how the
  tests were selected.}

\begin{enumerate}

\item{test-id1\\}

Type: \wss{Functional, Dynamic, Manual, Automatic, Static etc. Most will
  be automatic}
					
Initial State: 
					
Input: 
					
Output: \wss{The expected result for the given inputs}

Test Case Derivation: \wss{Justify the expected value given in the Output field}

How test will be performed: 
					
\item{test-id2\\}

Type: \wss{Functional, Dynamic, Manual, Automatic, Static etc. Most will
  be automatic}
					
Initial State: 
					
Input: 
					
Output: \wss{The expected result for the given inputs}

Test Case Derivation: \wss{Justify the expected value given in the Output field}

How test will be performed: 

\item{...\\}
    
\end{enumerate}

\subsubsection{Module 2}

...

\subsection{Tests for Nonfunctional Requirements}

\wss{If there is a module that needs to be independently assessed for
  performance, those test cases can go here.  In some projects, planning for
  nonfunctional tests of units will not be that relevant.}

\wss{These tests may involve collecting performance data from previously
  mentioned functional tests.}

\subsubsection{Module ?}
		
\begin{enumerate}

\item{test-id1\\}

Type: \wss{Functional, Dynamic, Manual, Automatic, Static etc. Most will
  be automatic}
					
Initial State: 
					
Input/Condition: 
					
Output/Result: 
					
How test will be performed: 
					
\item{test-id2\\}

Type: Functional, Dynamic, Manual, Static etc.
					
Initial State: 
					
Input: 
					
Output: 
					
How test will be performed: 

\end{enumerate}

\subsubsection{Module ?}

...

\subsection{Traceability Between Test Cases and Modules}

\wss{Provide evidence that all of the modules have been considered.}
				
\bibliographystyle{plainnat}

\bibliography{../../refs/References}

\newpage

\section{Appendix}

This is where you can place additional information.

\subsection{Symbolic Parameters}

The definition of the test cases will call for SYMBOLIC\_CONSTANTS.
Their values are defined in this section for easy maintenance.

\subsection{Usability Survey Questions?}

\wss{This is a section that would be appropriate for some projects.}


\end{document}