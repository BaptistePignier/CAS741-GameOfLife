\documentclass{article}

\usepackage{tabularx}
\usepackage{booktabs}

\title{\progname}

\author{\authname}

\date{01/19/2025}

%% Comments

\usepackage{color}

\newif\ifcomments\commentstrue %displays comments
%\newif\ifcomments\commentsfalse %so that comments do not display

\ifcomments
\newcommand{\authornote}[3]{\textcolor{#1}{[#3 ---#2]}}
\newcommand{\todo}[1]{\textcolor{red}{[TODO: #1]}}
\else
\newcommand{\authornote}[3]{}
\newcommand{\todo}[1]{}
\fi

\newcommand{\wss}[1]{\authornote{blue}{SS}{#1}} 
\newcommand{\plt}[1]{\authornote{magenta}{TPLT}{#1}} %For explanation of the template
\newcommand{\an}[1]{\authornote{cyan}{Author}{#1}}

%% Common Parts

\newcommand{\progname}{Problem Statement and Goals \\ Game of continuous life} % PUT YOUR PROGRAM NAME HERE
\newcommand{\authname}{Baptiste Pignier} % AUTHOR NAMES                  

\usepackage{hyperref}
    \hypersetup{colorlinks=true, linkcolor=blue, citecolor=blue, filecolor=blue,
                urlcolor=blue, unicode=false}
    \urlstyle{Same}
                                


\begin{document}

\maketitle

\section{Problem Statement}

This software is a continuous version of the Conway's Game of Life.
Space, time and rules are made continuous to explore more emerging phenomena.
This simulation allows the observation of cellular automata and their macroscopic evolutions, in a world governed by microscopic rules.
\subsection{Problem}

Make space, time and rules continuous is a difficult challenge and many extensions are possibles.
The problem is to define the framework of the simulation and to highlight its limits.

\subsection{Inputs and Outputs}

The inputs are many numerical parameters that define how the simulation's evolutions. The radius of the circle which defines the neighborhood of a cell or the sensitivity of a 
cell to the overcrowding of its neighborhood are examples of numerical parameters.  Each parameter is set before the simulation and their values define the rules of the simulation.
The output is a succession of environmental state in which we can appreciate the successive applications of the rules of evolutions.
Production is a statement of state of the environment in which we can assess the successive applications of the rules of evolution. 
These states will be displayed graphically in a graphic window.

\subsection{Stakeholders}

This project is intended for anyone interested in cellular automaton simulation.
This project is not intended to be a research tool for cellular automata because 
it will be much less complete than other already existing projects

\subsection{Environment}

The software requires a large amount of hardware resources to calculate the successive applications of the rules as well as the display of the result.
The display of the result will be delegated to optimized graphics libraries. Thus, the graphics implementation is not part of the software design. 
The operating systems supported will be those supported by the graphic librarie used, which will be chosen to support as many operating systems as possible

\section{Goals}

The main goal is a functionnal simulation and the faithful representation of the impact of the parameters on the simulation.
The original Conway's Game of Life must be able to be simulated by a the simpliest configuration of the parameters.

\section{Stretch Goals}

These goal can be extended with algorithms for parameter research resulting in a visually interesting simulation.

\section{Challenge Level and Extras}

Expected difficulty level is general and intermediate. I have knowledge on the subject of simulation design in general but little on the physical and mathematical aspect of the algorithms that I will implement. 

The simulation will be designed to be modular. Thus, extensions can be easily added to complicate the evolution rules.
  

\section{Inspirations}

\begin{itemize}
    \item \href{https://sourceforge.net/projects/smoothlife/}{The SmoothLife project}  
    \item \href{http://arxiv.org/abs/1111.1567}{The research paper}    
    \item \href{https://fr.wikipedia.org/wiki/Lenia}{Wikipedia page of another automata cellular} 
\end{itemize}

\end{document}